\chapter{Introduction}
The DNA – Dynamic Network Analyzer is a framework for Graph-theoretic Analysis of Dynamic Networks. In order to illustrate and compare generated data, DNA is equipped with numerous features to plot data in a sophisticated way, while still allowing the user to control the plotting process and to customize the plots visual representation aswell as creating his own custom plots.

Throughout this manual the numerous plotting features will be described aswell as their respective use explained. Examples give the user an insight on the impact of certain customizations.

The manual is divided into two parts. The first addresses the plotting process in general, what methods to choose and what parameters to set in order to reach the desired results. The second part demonstrates how to create plots, what possibilities and limits there are and how to handle custom plots in general.