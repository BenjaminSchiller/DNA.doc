\chapter{Configuration}
The graph visualization user interface is designed to be intuitive and easy to use. For most use cases the default configuration will offer a convenient GUI. However, more sophisticated occasions may require to adapt the interface to specific needs. This is achieved by defining custom ToolTips, GraphStyleRules and custom configuration files via JSON \cite{json}. The following sections will show how to work with the interface and create individual configurations.

\section{ToolTips}
\label{s:impl-tooltips}
The abstract ToolTip-class (\textit{dna.visualization.graph.toolTips.ToolTip}) is a wrapper class for the GraphStream Sprite class. Sprites are objects inside the GraphStream graph that can be used to display information and data. ToolTip introduces several methods, which make the handling of sprites more enjoyable. There are two types of ToolTips currently implemented in DNA: Buttons and InfoLabels. They are both represented by an abstract class (\textit{dna.visualization.graph.toolTips.button.Button} \& \textit{dna.visualization.graph.toolTips.infoLabel.InfoLabel}), which extend the basic ToolTip class. In order to implement own ToolTips one is encouraged to extend either Button or InfoLabel. Additionally, it is necessary to define a constructor as follows:
\begin{lstlisting}
public constructor(Sprite s, String name, Node n, Parameter[] params)
\end{lstlisting}
It will be called on initialization with exactly this parameters. The Parameter[] array can be used to pass additional parameters. Beside the constructor each ToolTip implementation (in this example the class would be "BananaToolTip") has to have a method:
\begin{lstlisting}
public static BananaToolTip getFromSprite(Sprite s) {
	return (BananaToolTip) ToolTip.getToolTipFromSprite(s);
}
\end{lstlisting}
For additional functionality, like mouse interactions, one can overwrite the \textit{onLeftClick()} and \textit{onRightClick()} methods. See the actual default ToolTip implementations for well documented examples. Also note that the \emph{dna.visualization.graph.rules.GraphStyleUtils}-class offers a variety of utility methods, which can be used to set labels, change sizes and colors and more.


\section{GraphStyleRules}
\label{s:graphStyleRules}
The GraphStyleRules are rules designed to change the graphical representation of the given graph. All GraphStyleRules have to extend the abstract \emph{GraphStyleRule} class (located in dna.visualization.graph.rules) and can override one or more of the following methods in order to dynamically behave and react to certain events:
\begin{itemize}
\item onNodeAddition(Node n, Weight w)
\item onNodeRemoval(Node n)
\item onNodeWeightChange(Node n, Weight wNew, Weight wOld)
\item onEdgeAddition(Edge e, Weight w, Node n1, Node n2)
\item onEdgeRemoval(Edge e, Node n1, Node n2)
\item onEdgeWeightChange(Edge e, Weight wNew, Weight wOld)
\end{itemize}
The handed over \emph{Node} and \emph{Edge} objects are nodes and edges of the graphstream graph (not from DNA!). Therefore, one can make use of the huge styling possibilities included in graphstream. The utility class \emph{GraphStyleUtils} contains several methods, e.g. \emph{setShape()}, \emph{setColor()} ..., which can be used to change the graphic presentation of the given nodes and edges. For example implementations check the included rules like \emph{RandomNodeSize} or \emph{RandomNodeColor}. In section \ref{ss:rulesConfig} we will discuss how GraphStyleRules can be loaded and used in gvis.


\section{GraphPanelConfig}
The GraphPanelConfig is a configuration object for a GraphPanel. It contains the most general configurable aspects as well as additional configuration objects, namely a \emph{WaitConfig}, \emph{ToolTipsConfig}, \emph{RulesConfig}, \emph{ProjectionConfig} and a \emph{CaptureConfig}. The general configuration structure is shown in figure \ref{fig:configstruct}.

\begin{figure} [h]
\centering
\includegraphics [scale=0.85] {images/configstructure.pdf}
\caption{Structure of the GraphPanelConfig and subconfigs.}
\label{fig:configstruct}
\end{figure}

A default GraphPanelConfig will be loaded if no other config path is given explicitly prior to initialization. This may be achieved either by setting the respective property via 
\begin{lstlisting}
	Config.overwrite("GRAPH_VIS_CONFIG_PATH", <config-path>);
\end{lstlisting}
or by calling the method:
\begin{lstlisting}
	GraphVisualization.setConfigPath(<config-path>);
\end{lstlisting}
All configurable values may be altered during runtime with getter- and setter-methods. In order to change values of subconfigs, e.g. the video record destination directory, one has to retrieve the CaptureConfig from the GraphPanelConfig, which is first retrieved from the GraphPanel directly. This specific case may look like this:
\begin{lstlisting}
	GraphPanel panel = GraphVisualization.getCurrentGraphPanel();
	GraphPanelConfig config = panel.getGraphPanelConfig();
	CaptureConfig ccfg = config.getCaptureConfig();
	ccfg.setVideoDir("myvideos/");
\end{lstlisting}
However, specific values, which are only read and used at initialization, won't have any effect upon change. E.g. the window size of the graph panel is only read when the panel is being initialized. Any modification at a later point won't affect the GUI. 
All configurable values of the GraphPanelConfig and their descriptions are listed in \ref{tab:graphPanelConfigValues}.

\begin{table}[h]
\caption{Configuration values of GraphPanelConfig.}
\centering
\begin{tabular}[h]{|l|l|l|}\hline
	\textbf{Value} & \textbf{Type} & \textbf{Description}\\
	\hline
	Width & Integer & Width of the visualization window.\\
	\hline
	Height & Integer & Height of the visualization window.\\
	\hline
	Fullscreen & Boolean & If true the visualization window\\
	& & will be maximized on initialization.\\
	\hline
	StatPanelEnabled & Boolean & If true the visualization window\\
	& & will contain the StatPanel.\\
	\hline
	TextPanelEnabled & Boolean & If true the visualization window\\
	& & will contain the TextPanel.\\
	\hline
	TimetampFormat & String & Sets the used timestamp format.\\
	\hline
	TimestampOffset & Long & Sets a offset to be added to the shown timestamp.\\
	\hline
	TimestampInSeconds & Boolean & If true the batch timestamp will be interpreted as seconds instead of milliseconds.\\
	\hline
	RenderHQ & Boolean & Enables high-quality rendering.\\
	\hline
	RenderAA & Boolean & Enables anti-aliasing.\\
	\hline
	ZoomSpeed & Double & Zoom speed factor.\\
	\hline
	ScrollSpeed & Double & Scroll speed factor.\\
	\hline
	RotationSpeed & Double & Rotation speed factor.\\
	\hline
	DirectedEdgeArrowsEnabled & Boolean & If true directed edges will\\
	& & be rendered with arrow-heads.\\
	\hline
	NodeSize & Double & Default size of nodes.\\
	\hline
	NodeColor & String & Default color of nodes. All CSS color-\\
	& & keywords are available \cite{css-colors}.\\
	\hline
	EdgeSize & Double & Default size of edges.\\
	\hline
	Layouter & Enum & Layouter used to layout the graph.\\
	& & Possible values: none, auto, linlog.\\
	\hline
	AutoLayoutForce & Double & Force of the auto-layouter.\\
	\hline
	WaitConfig & JSON-Object & A wait config. See section \ref{ss:waitConfig}.\\
	\hline
	RulesConfig & JSON-Object & A rules config. See section \ref{ss:rulesConfig}.\\
	\hline
	ProjectionConfig & JSON-Object & A Projection config. See section \ref{ss:projectionConfig}.\\
	\hline
	CaptureConfig & JSON-Object & A capture config. See section \ref{ss:captureConfig}.\\
	\hline
\end{tabular}
\label{tab:graphPanelConfigValues}
\end{table}

\subsection{WaitConfig}
\label{ss:waitConfig}
The WaitConfig object is used to configure special wait times used for smooth graph visualization. When adding multiple nodes and edges at once the layout algorithm may cause a very shaking and unstable view of the graph. Therefore, artificial wait times after each event are used to guarantee a smoother experience. Table \ref{tab:waitConfigValues} shows all configurable values.

\begin{table}[h]
\caption{Configuration values of WaitConfig.}
\centering
\begin{tabular}[h]{|l|l|l|}\hline
	\textbf{Value} & \textbf{Type} & \textbf{Description}\\
	\hline
	Enabeld & Boolean & If true wait times are enabled.\\
	\hline
	EdgeAddition & Long & Wait time after edge additions in ms.\\
	\hline
	EdgeRemoval & Long & Wait time after edge removals in ms.\\
	\hline
	EdgeWeightChange & Long & Wait time after edge weight changes in ms.\\
	\hline
	NodeAddition & Long & Wait time after node additions in ms.\\
	\hline
	NodeRemoval & Long & Wait time after node removals in ms.\\
	\hline
	NodeWeightChange & Long & Wait time after node weight changes in ms.\\
	\hline
\end{tabular}
\label{tab:waitConfigValues}
\end{table}

\subsection{ToolTipsConfig}
\label{ss:toolTipsConfig}
In order to define which ToolTips will be displayed and how they are ordered and initialized one can use ToolTipsConfigs. The default ToolTipsConfig looks as follows:
\begin{lstlisting}
"ToolTipsConfig": {
	"Enabled": true,
	"dna.visualization.graph.toolTips.infoLabel.NodeDegreeLabel": {
		"Name": "Degree",
		"Index": 1
	},
	"dna.visualization.graph.toolTips.infoLabel.NodeIdLabel": {
		"Name": "Node",
		"Index": 0
	},
	"dna.visualization.graph.toolTips.button.FreezeButton": {
		"Name": "FreezeButton",
		"Index": 2
	},
	"dna.visualization.graph.toolTips.button.HighlightButton": {
		"Name": "HighlightButton",
		"Growth": 15.0,
		"Index": 3
	}
}
\end{lstlisting}
The ToolTipsConfig only has one static config field, which is \emph{Enabled} and can be set true or false in order to enable/disable ToolTips. Additionally, all used ToolTips can be defined using their respective class-path as a key. The contained \emph{Name} field will be used to uniquely identify the ToolTips. The \emph{Index} field is used to set the order in which ToolTips will be shown in the visualization. All additional defined values (e.g. \emph{Growth} in the HighlightButton definition) will be handed over to the ToolTip constructor as parameters, see section \ref{s:impl-tooltips} for more details about the implementation.


\subsection{RulesConfig}
\label{ss:rulesConfig}
The graph visualization allows to define so called \emph{GraphStyleRules} to modify the graph presentation in several ways. See section \ref{s:graphStyleRules} for more details. In this section we focus on how to enable and configure said rules using the configuration. A RulesConfig holds multiple configurable \emph{GraphStyleRuleConfis}, each representing a single GraphStyleRule. Each GraphStyleRuleConfig will be identified by a key pointing to the rules class-path. Additionally, there are three shared configurable values for each rule: the \emph{name}, the \emph{enabled}-flag and the \emph{hidden}-flag. Furthermore, one may add additional parameters, which will be used by the respective rules. Note: The way these parameters will be treated is up to the given rule. See the following example \ref{config:exampleRuleConfig} for an actual RulesConfig definition. The first rule contains all possible configurable values and one additional parameter: \emph{Growth}. The NodeSizeByDegree class increases/decreases each nodes size by the growth-value whenever their degree is being incremented or decremented respectively. The second rule does not contain any values at all. However it will still be enabled.

\begin{figure} [h]
\begin{lstlisting}
"RulesConfig": {
	"dna.visualization.graph.rules.nodes.NodeSizeByDegree": {
		"Name": "defaultNodeSizeByDegree",
		"Growth": 0.3,
		"Enabled": true,
		"Hidden": false,
	},
	"dna.visualization.graph.rules.nodes.RandomNodeColor": {},
}
\end{lstlisting}
\caption{Example RuleConfig definition.}
\label{config:exampleRuleConfig}
\end{figure}

\begin{table}[h]
\caption{Configuration values of a GraphStyleRuleConfig.}
\centering
\begin{tabular}[h]{|l|l|l|}\hline
	\textbf{Value} & \textbf{Type} & \textbf{Description}\\
	\hline
	Name & String & Name of the rule.\\
	& & (If not set \emph{key} will be used as name.)\\
	\hline
	Enabled & Boolean & If set the rule is enabled.\\
	& & (If not set will be assumed as \emph{true}.)\\
	\hline
	Hidden & Boolean & If set the rule is hidden from the interface.\\
	& & (If not set will be assumed as \emph{false}.)\\
	\hline
	Parameter$_0$ & Parameter & First parameter of the respective rule.\\
	\hline
	Parameter$_1$ & Parameter & Second parameter of the respective rule.\\
	\hline
	... & ... & ...\\
	\hline	
	Parameter$_i$ & Parameter & i-th parameter of the respective rule.\\
	\hline
\end{tabular}
\label{tab:graphStyleRuleConfigValues}
\end{table}


\subsection{ProjectionConfig}
\label{ss:projectionConfig}
Since the graphstream library does not support three-dimensional coordinates, two simple projection mechanisms have been implemented to simulate 3D behaviour. In this mode the node weights are assumed as 3D coordinates. For all configurable values see table \ref{tab:projectionConfigValues}.

\subsubsection{Vanishing Point Projection}
This projection mode uses a defined vanishing point and virtually \emph{shifts} all nodes towards the vanishing point. How far the nodes will be shifted depends on the respective $z$-coordinate. The further a point is away from the screen-plain ($z=0$) the further it will be shifted. The vanishing point coordinates, a scaling factor and logarithmic scaling can be configured in the ProjectionConfig. Note: If 3d-projection is enabled but vanishing point mode is disabled ortographic projection will be used instead.

\subsubsection{Ortographic Projection}
The ortographic projection uses a transformation matrix to project the three-dimensional nodes onto the two-dimensional screen-plain, see figure \ref{math:orthographicProjection}. The used matrix and a offset may be defined in the ProjectionConfig. Note: This mode is only used if the vanishing point projection is disabled.

\begin{figure} [h]
\[
\left(
\begin{array}{c}
n_{x}^{'} \\
n_{y}^{'}
\end{array}
\right)
%
=
%
\left(
\begin{array}{ccc}
x_0 & y_0 & z_0 \\
x_1 & y_1 & z_1
\end{array}
\right)
\cdot
\left(
\begin{array}{c}
n_x  \\
n_y  \\
n_z
\end{array}
\right)
%
+
%
\left(
\begin{array}{c}
offset_x \\
offset_y
\end{array}
\right)
\]
\caption{Ortographic projection from node $n$ to two-dimensional node $n^{'}$.}
\label{math:orthographicProjection}
\end{figure}

\begin{table}[h]
\caption{Configuration values of a ProjectionConfig.}
\centering
\begin{tabular}[h]{|l|l|l|}\hline
	\textbf{Value} & \textbf{Type} & \textbf{Description}\\
	\hline
	Enabled & Boolean & If true 3d projection will be used.\\
	\hline
	UseVanishingPoint & Boolean & If true vanishing point will be used\\
	\hline
	VanishingPointLogScale & Boolean & If true the projection towards the\\
	& & VP will be scaled logarithmic.\\
	\hline
	VanishingPoint\textunderscore X & Double & $x$-coordinate of the VP.\\
	\hline
	VanishingPoint\textunderscore Y & Double & $y$-coordinate of the VP.\\
	\hline
	VanishingPoint\textunderscore Z & Double & $z$-coordinate of the VP.\\
	\hline
	VanishingPointScaling & Double & Weight of the projection scaling.\\
	\hline
	ScalingMatrixS0\textunderscore X & Double & $x_0$ value of the scaling matrix.\\
	& & For ortographic use only.\\
	\hline
	ScalingMatrixS0\textunderscore Y & Double & $y_0$ value of the scaling matrix.\\
	& & For ortographic use only.\\
	\hline
	ScalingMatrixS0\textunderscore Z & Double & $z_0$ value of the scaling matrix.\\
	& & For ortographic use only.\\
	\hline
	ScalingMatrixS1\textunderscore X & Double & $x_1$ value of the scaling matrix.\\
	& & For ortographic use only.\\
	\hline
	ScalingMatrixS1\textunderscore Y & Double & $y_1$ value of the scaling matrix.\\
	& & For ortographic use only.\\
	\hline
	ScalingMatrixS1\textunderscore Z & Double & $z_1$ value of the scaling matrix.\\
	& & For ortographic use only.\\
	\hline
	OffsetVector\textunderscore X & Double & $x$-value of the offset vector.\\
	& & For ortographic use only.\\
	\hline
	OffsetVector\textunderscore Y & Double & $y$-value of the offset vector.\\
	& & For ortographic use only.\\
	\hline
\end{tabular}
\label{tab:projectionConfigValues}
\end{table}

\subsection{CaptureConfig}
\label{ss:captureConfig}
The capture config defines how screenshot and video captures will behave. This includes basic options like destination directory as well as more detailed options, e.g. the frames-per-second included in the video file. For all available configuration values see table \ref{tab:captureConfigValues}.

\begin{table}[h]
\caption{Configuration values of CaptureConfig.}
\centering
\begin{tabular}[h]{|l|l|l|}\hline
	\textbf{Value} & \textbf{Type} & \textbf{Description}\\
	\hline
	RecordArea & Enum & Default area to be recorded.\\
	& & Possible values: content, graph, full\\
	\hline
	VideoDir & String & Destination directory for captured videos.\\
	\hline
	VideoSuffix & String & Suffix of video files.\\
	\hline
	VideoFilename & String & Filename of captured video files. If set as\\
	& & "null" names will be derived from the graph-\\
	& & panel name and (real-time) timestamp.\\
	\hline
	VideoMaximumLength & Integer & Maximum length of videos in seconds.\\
	\hline
	VideoRecordFps & Integer & Frames per second recorded from the screen.\\
	\hline
	VideoFileFps & Integer & Frames per second of the resulting video file.\\
	\hline
	VideoAutoRecord & Boolean & If set video recording will automatically\\
	& & start upon initialization.\\
	\hline
	VideoUseCodec & Boolean & If set video will be encoded using\\
	& & the Techsmith codec \cite{techsmith-codecs}. Else uncompressed.\\
	\hline
	VideoBufferImagesOnFs & Boolean & If true images will be buffered on the filesystem.\\
	\hline
	ScreenshotDir & String & Destination directory for screenshots.\\
	\hline
	ScreenshotFormat & String & Screenshot format (jpeg, png, bmp, wbmp, gif).\\
	\hline
	ScreenshotStabilityThreshold & Double & Stability threshold of the layouter when\\
	& & taking screenshots waiting for stability.\\
	\hline
	ScreenshotStabilityTimeout & Long & Stability timeout for taking screenshots\\
	& & waiting for stability in ms.\\
	\hline
	ScreenshotForegroundDelay & Long & Artificial delay when taking screenshots to\\
	& & let the machine move the panel to foreground.\\
	\hline
\end{tabular}
\label{tab:captureConfigValues}
\end{table}

\section{Default-Configuration file}
The default configuration file located in ”config/gvis/gvis default.cfg” holds default values for each type of object and all possible properties. It represents the ultimate structure a configuration file can assume. When creating manual configuration files missing values will be replaced with the respective values from the default configuration, thus allowing to craft own configurations easily and with little effort. Note that any changes on the default file can prevent the interface from running, for example removing values that are needed as default references.

\section{Building own configuration files}
When creating an own configuration file it is recommended to take the default config structure and fill it with all desired values and settings. The following example \ref{config:randomNodes} shows a minimal configuration with the random node sizes and colors. Additionally the size is set to 640x480 and the stat- and text-panels are disabled. See figure \ref{fig:randomNodesVis} for the resulting GraphPanel.

\begin{figure} [h]
\begin{lstlisting}
{
	"GraphPanelConfig": {
		"Width": 640, 
		"Height": 480,
	
		"StatPanelEnabled": false,
		"TextPanelEnabled": false,
	
		"RulesConfig": {	
			"dna.visualization.graph.rules.nodes.RandomNodeSize": {},
			"dna.visualization.graph.rules.nodes.RandomNodeColor": {},
		}
	}
}
\end{lstlisting}
\caption{Example of a small GraphPanelConfig.}
\label{config:randomNodes}
\end{figure}

\begin{figure} [h]
\centering
\includegraphics [scale=0.9] {images/random-node-example}
\caption{Example visualization of the GraphPanelConfig from \ref{config:randomNodes}.}
\label{fig:randomNodesVis}
\end{figure}

