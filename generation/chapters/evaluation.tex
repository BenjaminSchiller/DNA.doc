\chapter{Evaluation}
In order to evaluate the data and reach certain conclusions from the results it is neccessary to take a view at the data directly or atleast to visualize them in a plot or graph. For this purpose DNA offers sophisticated plotting mechanisms, automatic LaTeX output, which includes data tables and plots, and a graphical user interface namely the DNA-Visualizer.

\subsection{Aggregation}
The aggregation process is enabled by default and takes place at the end of the generation process. During aggregation the data from all runs will be aggregated and stored in a separate aggregation directory \textit{aggr/} next to the run directories. The aggregations filesystem structure is analogous to the structure of a single run. The only difference is the extended data representation. All values contain aggregated \textit{average}, \textit{minimum}, \textit{maximum}, \textit{median}, \textit{variance}, \textit{variance\textunderscore low}, \textit{variance\textunderscore up}, \textit{confidence\textunderscore low} and \textit{confidence\textunderscore up} in this order. A simple example follows:

A series is generated with 2 runs and 100 batches. The metric-value \textit{degreeMax} has the its values shown in table \ref{tab:degreemax-data} and its aggregated data in table \ref{tab:degreemax-aggr} respectively.

\begin{table}[h]
\centering
\begin{tabular}[h]{|l|l|l|}\hline
	\textbf{degreeMax} & &\\
	\hline
	\textbf{batch} & \textbf{run.0} & \textbf{run.1}\\
	\hline
	0 & 13 & 15\\
	\hline
	1 & 12 & 16\\
	\hline
	2 & 12 & 15\\
	\hline
	... & ... & ... \\
	\hline
\end{tabular}
\caption{Data values of \textit{degreeMax}.}
\label{tab:degreemax-data}
\end{table}
\begin{table}[h]
\centering
\begin{tabular}[h]{|l|l|l|l|l|l|l|l|l|l|}\hline
	\textbf{batch} & \textbf{avg} & \textbf{min} & \textbf{max} & \textbf{med} & \textbf{var} & \textbf{var-} & \textbf{var+} & \textbf{conf-} & \textbf{conf+}\\
	\hline
	0 & 14.0 & 13.0 & 15.0 & 15.0 & 1.0 & 1.0 & 1.0 & 5.013 & 22.987\\
	\hline
	1 & 14.0 & 12.0 & 16.0 & 16.0 & 4.0 & 4.0 & 4.0 & -3.975 & 31.975\\
	\hline
	2 & 13.5 & 12.0 & 15.0 & 15.0 & 2.25 & 2.25 & 2.25 & 0.019 & 26.981\\
	\hline
	... & ... & ... & ... & ... & ... & ... & ... & ... & ...\\
	\hline
\end{tabular}
\caption{Aggregated values of \textit{degreeMax}.}
\label{tab:degreemax-aggr}
\end{table}



\subsection{Plotting}
TODO: Plotting

\subsection{LaTeX}
TODO: LaTeX

\subsection{Visualization}
TODO: Visualization