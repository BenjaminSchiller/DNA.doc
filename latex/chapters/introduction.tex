\chapter{Introduction}
The DNA – Dynamic Network Analyzer is a framework for graph-theoretic Analysis of Dynamic Networks. The built-in LaTeX-output feature allows to illustrate and compare data by generating LaTeX-documents containing data and plots.

This manual will show how DNA can be used to generate LaTeX-documents aswell as what options and parameters may be utilized in order to customize the output. Examples are going to be shown to give the user an insight on the impact of certain customizations.


%The manual is divided into two parts. The first addresses the plotting process in general, what %methods to choose and what parameters to set in order to reach the desired results. The %second part demonstrates how to create plots, what possibilities and limits there are and how %to handle custom plots in general.