\chapter{Commandline-Parameters}
The program can be started with different commandline-parameters. The syntax is as follows:\
\center{java --jar dna--vis.jar \quad [--c \textless config--path\textgreater ] [--d \textless data--dir\textgreater ] [--h] [--l] [--p] [--z] [--zr]}\


\begin{table}[h]
\centering
\begin{tabular}[h]{|l|l|}\hline
	\textbf{Parameter} & \textbf{Description}\\
	\hline
	--c \textless config--path\textgreater & Used to give the program a path to  own configuration file.\\
	& If left out, the default-config will be used. \\
	\hline
	--d \textless data--dir\textgreater & Overwrites the default data-directory given in \\
	& the configuration file. \\
	\hline
	--h & Displays the help message. \\
	\hline
	--l & Launches the GUI in Livedisplay-Mode.\\
	\hline
	--p & Launches the GUI in Playback-Mode.\\
	\hline
	--z & Enables zipped batches.\\
	\hline
	--zr & Enables zipped runs. Only available in PlayBack mode!\\
	\hline
\end{tabular}
\caption{Explanation of the commandline-parameters.}
\label{tab:cmdline1}
\end{table}

\section{Examples}
\begin{flushleft}
java --jar dna--vis.jar --h\\
java --jar dna--vis.jar --c "config/min\textunderscore cfg.cfg"\\
java --jar dna--vis.jar --l --d "data/test1337/run.5/" --z\\
java --jar dna--vis.jar --d "data/dd/run.0"\\
java --jar dna--vis.jar --c "config/gui\textunderscore min.cfg" --d "data/cc/run.42/" --p --z\\
\end{flushleft}